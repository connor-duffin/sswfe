\documentclass[11pt]{article}

\usepackage{amsmath}
\usepackage{amssymb}
\usepackage{amsthm}
\usepackage[margin=1in]{geometry}
\usepackage{graphicx}
\usepackage{subcaption}
\usepackage[round]{natbib}

\usepackage{newpxtext}
\usepackage{inconsolata}

\newtheorem{theorem}{Theorem}

\input{/Users/connor/Documents/LaTeX/macros.tex}

\title{Weak form for SWE}
\author{Connor Duffin}

\begin{document}

\maketitle

In this note we derive the weak form for the shallow water equations (SWE). The
shallow water equations, in the non-conservative form, are given by
\begin{gather*}
  \partial_t \mb{u} + \mb{u} \cdot \nabla \mb{u} + g \nabla h - \nu \nabla \cdot \mb{T} +
  C_D \frac{\lVert \mb{u} \rVert_2 \mb{u}}{H + h} = 0, \\
  \partial_t h + \nabla \cdot \left( (H + h) \mb{u} \right) = 0,
\end{gather*}
where $\mb{u} := \mb{u}(x, y, t) = (u_1(x, y, t), u_2(x, y, t))$, $h := h(x, y, t)$.
We assume that $(x, y) \in \Omega$, with boundary $\Gamma$. These equations are
of course supplemented with the appropriate boundary conditions, specifying
behaviour on $\Gamma$. The parameters are given by
\begin{itemize}
\item $g = 9.8$, the acceleration due to gravity.
\item $H$, the mean surface height.
\item $C_D$, the friction coefficient.
\item $\nu$, the kinematic viscosity.
\item $\mb{T} = \nabla {\mb{u}} + \nabla{\mb{u}}^\top
  - \frac{2}{3}\left(\nabla \cdot \mb{u} \right) \mb{I}$, the viscous stress
  tensor for the (assumed) Newtonian fluid.
\end{itemize}

Next we dot each equation by a test function, $\mb{v}(x, y) =
(v_1(x, y), v_2(x, y))$, and integrate over the problem domain
$\Omega$. This gives
\begin{gather*}
  \int_\Omega \partial_t \mb{u} \cdot \mb{v} \, \dee \mb{x}
  + \int_\Omega \left(\mb{u} \cdot \nabla \mb{u}\right) \cdot \mb{v} \, \dee \mb{x}
  + \int_\Omega g \nabla h \cdot \mb{v} \, \dee \mb{x}
  - \int_\Omega \nu \left(\nabla \cdot \mb{T}\right) \cdot \mb{v} \, \dee \mb{x}
  + \int_\Omega C_D \frac{\lVert \mb{u} \rVert_2}{H + h} \mb{u} \cdot \mb{v} \, \dee \mb{x} = 0
\end{gather*}

To reduce this further we need two results from~\cite{gonzalez2008first}. We
state these here for completeness.

\begin{theorem}
  \label{thm:tensor-product-rule}
  Suppose $\mb{S}$ and $\mb{a}$ are a second-order tensor field and a vector
  field, respectively. Then
  \[
    \nabla \cdot (\mb{S}^\top \mb{a}) =
    (\nabla \cdot \mb{S}) \cdot \mb{a} + \mb{S} : \nabla \mb{a},
  \]
  where $:$ denotes the tensor reduction operation.
\end{theorem}

\begin{theorem}
  \label{thm:divergence}
  (Divergence theorem) Let $\mb{a}$ be some arbitrary vector field which is
  defined over a region $\Omega$, a subset set of $\mathbb{R}^2$. Then
  \[
    \int_\Omega \nabla \cdot \mb{a} \, \dee \mb{x} = 
    \int_\Gamma \mb{a} \cdot \mb{n} \, \dee s.
  \]
\end{theorem}

The only term which we will integrate by parts is the viscous stress term, as is
standard in FEM. This gives, via the Theorems provided above:
\begin{align*}
  \int_\Omega \nu \left(\nabla \cdot \mb{T}\right) \cdot \mb{v} \, \dee \mb{x}
  &= \nu \left( \int_\Omega \nabla \cdot (\mb{T}^\top \mb{v}) - \mb{T} : \nabla \mb{v} \, \dee \mb{x} \right) \\
  &= \nu \left(\int_\Gamma (\mb{T}^\top \mb{v}) \cdot \mb{n} \, \dee s
    - \int_\Omega \mb{T} : \nabla \mb{v} \, \dee \mb{x} \right) \\
  &= \nu \int_\Gamma \mb{v} \cdot \mb{T} \mb{n} \, \dee s
    - \nu \int_\Omega \mb{T} : \nabla \mb{v} \, \dee \mb{x}.
\end{align*}
Now, depending on which boundary conditions are imposed, the first term may drop
out. In~\citep{jacobs2015firedrakefluids} it is assumed that this stress tensor
is zero perpendicular to the boundary, so this term just drops out. However we
leave it in for now to have a full weak form for the system. This gives the full
weak form for the velocity components as
\begin{gather*}
  \int_\Omega \partial_t \mb{u} \cdot \mb{v} \, \dee \mb{x}
  + \int_\Omega \left(\mb{u} \cdot \nabla \mb{u}\right) \cdot \mb{v} \, \dee \mb{x}
  + \int_\Omega g \nabla h \cdot \mb{v} \, \dee \mb{x} \\
  + \int_\Omega \nu \mb{T} : \nabla \mb{v} \, \dee \mb{x}
  + \int_\Omega C_D \frac{\lVert \mb{u} \rVert_2}{H + h} \mb{u} \cdot \mb{v} \, \dee \mb{x} \\
  - \nu \int_\Gamma \mb{v} \cdot \mb{T} \mb{n} \, \dee s = 0
\end{gather*}

Now for the surface height we, again, multiply by a testing function $v(x, y)$
and integrate over the domain $\Omega$:
\[
  \int_\Omega \partial_t h v \, \dee \mb{x}
  + \int_\Omega \nabla \cdot \left( (H + h) \mb{u} \right) v \, \dee \mb{x} = 0.
\]
Depending on the prescribed boundary conditions this divergence term can be
integrated by parts
\[
  \int_\Omega \nabla \cdot \left( (H + h) \mb{u} \right) v \, \dee \mb{x}
  = - \int_\Omega \nabla v \cdot \left( (H + h) \mb{u} \right) \, \dee \mb{x}
  + \int_\Gamma v ((H + h) \mb{u}) \cdot \mb{n} \, \dee s = 0.
\]
giving the weak form for the surface height as
\[
  \int_\Omega \partial_t h v
   - \nabla v \cdot \left( (H + h) \mb{u} \right) \, \dee \mb{x}
  + \int_\Gamma v ((H + h) \mb{u}) \cdot \mb{n} \, \dee s = 0.
\]
Thus, the full system of the shallow water equations is:
\begin{gather*}
  \int_\Omega \partial_t \mb{u} \cdot \mb{v} \, \dee \mb{x}
  + \int_\Omega \left(\mb{u} \cdot \nabla \mb{u}\right) \cdot \mb{v} \, \dee \mb{x}
  + \int_\Omega g \nabla h \cdot \mb{v} \, \dee \mb{x} \\
  + \int_\Omega \nu \mb{T} : \nabla \mb{v} \, \dee \mb{x}
  + \int_\Omega C_D \frac{\lVert \mb{u} \rVert_2}{H + h} \mb{u} \cdot \mb{v} \, \dee \mb{x}
  - \nu \int_\Gamma \mb{v} \cdot \mb{T} \mb{n} \, \dee s = 0, \\
  \int_\Omega \partial_t h v
  - \nabla v \cdot \left( (H + h) \mb{u} \right) \, \dee \mb{x}
  + \int_\Gamma v ((H + h) \mb{u}) \cdot \mb{n} \, \dee s = 0.
\end{gather*}

\newpage

% \subsection*{Specific weak form: MMS}
% \begin{align*}
%   \int_\Omega
%   \partial_t u_1 v_1 +  \mb{u} \cdot \nabla u_1 v_1 +  g \partial_x h v_1
%   + C_D \frac{\lVert \mb{u} \rVert_2}{H + h} u_1 v_1
%   + \nu \nabla u_1 \cdot \nabla v_1 \, \dee \mb{x}
%   - \int_\Gamma v_1 \nabla u_1 \cdot \mb{n} \, \dee s &= 0, \\
%   \int_\Omega \partial_t u_2 v_2 +  \mb{u} \cdot \nabla u_2 v_2 +  g \partial_y h v_2
%   + C_D \frac{\lVert \mb{u} \rVert_2}{H + h} u_2 v_2
%   + \nu \nabla u_2 \cdot \nabla v_2 \, \dee \mb{x}
%   - \int_\Gamma v_2 \nabla u_2 \cdot \mb{n} \, \dee s &= 0, \\
%   \int_\Omega \partial_t h v_3
%   - \nabla v_3 \cdot \left( (H + h) \mb{u} \right) \, \dee \mb{x}
%   + \int_\Gamma v_3 ((H + h) \mb{u}) \cdot \mb{n} \, \dee s &= 0.
% \end{align*}
% Otherwise, if we just leave the divergence term as-is, we get
% \begin{align*}
%   \int_\Omega
%   \partial_t u_1 v_1 +  \mb{u} \cdot \nabla u_1 v_1 +  g \partial_x h v_1
%   + C_D \frac{\lVert \mb{u} \rVert_2}{H + h} u_1 v_1
%   + \nu \nabla u_1 \cdot \nabla v_1 \, \dee \mb{x}
%   - \int_\Gamma v_1 \nabla u_1 \cdot \mb{n} \, \dee s &= 0, \\
%   \int_\Omega \partial_t u_2 v_2 +  \mb{u} \cdot \nabla u_2 v_2 +  g \partial_y h v_2
%   + C_D \frac{\lVert \mb{u} \rVert_2}{H + h} u_2 v_2
%   + \nu \nabla u_2 \cdot \nabla v_2 \, \dee \mb{x}
%   - \int_\Gamma v_2 \nabla u_2 \cdot \mb{n} \, \dee s &= 0, \\
%   \int_\Omega \partial_t h v_3 \, \dee \mb{x}
%   + \int_\Omega \nabla \cdot \left( (H + h) \mb{u} \right) v_3 \, \dee \mb{x} &= 0.
% \end{align*}

% For the MMS verification, we have the unit square domain
% $\Omega = [0, 1] \times [0, 1]$. On the boundaries $\Gamma$ we set
% \begin{gather*}
%   \mb{u}(x, y, t) = (\cos(x)\sin(y), \sin(x^2) + \cos(y)), \\
%   h(x, y) = \sin(x)\sin(y),
% \end{gather*}
% the manufactured solutions for this example. Therefore all the surface integrals
% drop out as all test functions are identically zero on $\Gamma$. For this
% example we also integrate the divergence term by parts, so that the weak form is
% \begin{align*}
%   \int_\Omega
%   \partial_t u_1 v_1 +  \mb{u} \cdot \nabla u_1 v_1 +  g \partial_x h v_1
%   + C_D \frac{\lVert \mb{u} \rVert_2}{H + h} u_1 v_1
%   + \nu \nabla u_1 \cdot \nabla v_1 \, \dee \mb{x} &= 0, \\
%   \int_\Omega \partial_t u_2 v_2 +  \mb{u} \cdot \nabla u_2 v_2 +  g \partial_y h v_2
%   + C_D \frac{\lVert \mb{u} \rVert_2}{H + h} u_2 v_2
%   + \nu \nabla u_2 \cdot \nabla v_2 \, \dee \mb{x} &= 0, \\
%   \int_\Omega \partial_t h v_3
%   - \nabla v_3 \cdot \left( (H + h) \mb{u} \right) \, \dee \mb{x} &= 0.
% \end{align*}

% The next thing we do is discretise the system in time, to give an interative
% updating procedure. So far I have mainly considered an implicit method. Letting
% the solution at time $n \Dt$ be $u^n := u^n(x, y) = u(x, y, n \Dt)$, this gives
% \begin{align*}
%   \int_\Omega
%   \left(\frac{ u_1^n - u_1^{n - 1}}{\Dt} \right) v_1
%   + \mb{u}^{n - 1} \cdot \nabla u_1^n \, v_1
%   + g h_x^n \, v_1
%   + C_D \frac{\lVert \mb{u}^{n - 1} \rVert_2}{H + h^n} u_1^n \, v_1
%   + \nu \nabla u_1^n \cdot \nabla v_1 \, \dee \mb{x} &= 0, \\
%   \int_\Omega \left( \frac{u_2^n - u_2^{n - 1}}{\Dt} \right) v_2
%   + \mb{u}^{n - 1} \cdot \nabla u_2^n \, v_2
%   + g h_y^n \, v_2
%   + C_D \frac{\lVert \mb{u}^{n - 1} \rVert_2}{H + h^n} u_2^n \, v_2
%   + \nu \nabla u_2^n \cdot \nabla v_2 \, \dee \mb{x} &= 0, \\
%   \int_\Omega \left( \frac{h^n - h^{n - 1}}{\Dt} \right) v_3
%   - \nabla v_3 \cdot \left( (H + h^n) \mb{u}^n \right) \, \dee \mb{x} &= 0.
% \end{align*}

% \subsection*{Specific weak form: laminar flow in a channel}

% For the laminar flow in a channel, we consider the rectangular domain $\Omega =
% [0, d_x] \times [0, d_y]$. This gives us four boundaries: $\Gamma_L$,
% $\Gamma_R$, $\Gamma_T$, and $\Gamma_B$. These stand for the left, right, top and
% bottom boundaries. The BC's are:
% \[
%   \mb{u} =
%   \begin{cases}
%     (u_i, 0) & \mb{x} \in \Gamma_L \cup \Gamma_R, \\
%     (0, 0) & \mb{x} \in \Gamma_T \cup \Gamma_B,
%   \end{cases}
% \]
% So we have a specific inflow and outflow condition, and a no-slip condition on
% the top and bottom walls. On the $h$ component, there are no boundary
% conditions. Therefore, the weak form is
% \begin{align*}
%   \int_\Omega
%   \partial_t u_1 v_1 +  \mb{u} \cdot \nabla u_1 v_1 +  g \partial_x h v_1
%   + C_D \frac{\lVert \mb{u} \rVert_2}{H + h} u_1 v_1
%   + \nu \nabla u_1 \cdot \nabla v_1 \, \dee \mb{x}
%   - \int_\Gamma v_1 \nabla u_1 \cdot \mb{n} \, \dee s &= 0, \\
%   \int_\Omega \partial_t u_2 v_2 +  \mb{u} \cdot \nabla u_2 v_2 +  g \partial_y h v_2
%   + C_D \frac{\lVert \mb{u} \rVert_2}{H + h} u_2 v_2
%   + \nu \nabla u_2 \cdot \nabla v_2 \, \dee \mb{x}
%   - \int_\Gamma v_2 \nabla u_2 \cdot \mb{n} \, \dee s &= 0, \\
%   \int_\Omega \partial_t h v_3
%   - \nabla v_3 \cdot \left( (H + h) \mb{u} \right) \, \dee \mb{x}
%   + \int_{\Gamma_L} v_3 ((H + h) \mb{u}) \cdot \mb{n} \, \dee s
%   + \int_{\Gamma_R} v_3 ((H + h) \mb{u}) \cdot \mb{n} \, \dee s &= 0.
% \end{align*}
% Note that the surface integrals on the $\mb{u}$ weak forms have dropped out, but
% some of the surface integral terms from the $h$ component have remained in the
% system. These are on the inflow and outflow boundaries. As $\mb{u} = (0, 0)$ on
% $\Gamma_T$ and $\Gamma_B$, boundary integrals across these domains drop out of
% the system.

% When we discretise this system in time, we get:
% \begin{align*}
%   \int_\Omega
%   \left(\frac{ u_1^n - u_1^{n - 1}}{\Dt} \right) v_1
%   + \mb{u}^{n - 1} \cdot \nabla u_1^n \, v_1
%   + g h_x^n \, v_1
%   + C_D \frac{\lVert \mb{u}^{n - 1} \rVert_2}{H + h^n} u_1^n \, v_1
%   + \nu \nabla u_1^n \cdot \nabla v_1 \, \dee \mb{x} &= 0, \\
%   \int_\Omega \left( \frac{u_2^n - u_2^{n - 1}}{\Dt} \right) v_2
%   + \mb{u}^{n - 1} \cdot \nabla u_2^n \, v_2
%   + g h_y^n \, v_2
%   + C_D \frac{\lVert \mb{u}^{n - 1} \rVert_2}{H + h^n} u_2^n \, v_2
%   + \nu \nabla u_2^n \cdot \nabla v_2 \, \dee \mb{x} &= 0, \\
%   \int_\Omega \left( \frac{h^n - h^{n - 1}}{\Dt} \right) v_3
%   - \nabla v_3 \cdot \left( (H + h^n) \mb{u}^n \right) \, \dee \mb{x}
%   + \int_{\Gamma_L} v_3 ((H + h^n) \mb{u}^n) \cdot \mb{n} \, \dee s
%   + \int_{\Gamma_R} v_3 ((H + h^n) \mb{u}^n) \cdot \mb{n} \, \dee s &= 0.
% \end{align*}

% However, we can also use the weak form which does not integrate any terms by
% parts. In this instance we have
% \begin{align*}
%   \int_\Omega
%   \partial_t u_1 v_1 +  \mb{u} \cdot \nabla u_1 v_1 +  g \partial_x h v_1
%   + C_D \frac{\lVert \mb{u} \rVert_2}{H + h} u_1 v_1
%   + \nu \nabla u_1 \cdot \nabla v_1 \, \dee \mb{x} &= 0, \\
%   \int_\Omega \partial_t u_2 v_2 +  \mb{u} \cdot \nabla u_2 v_2 +  g \partial_y h v_2
%   + C_D \frac{\lVert \mb{u} \rVert_2}{H + h} u_2 v_2
%   + \nu \nabla u_2 \cdot \nabla v_2 \, \dee \mb{x} &= 0, \\
%   \int_\Omega \partial_t h v_3 \, \dee \mb{x}
%   + \int_\Omega \nabla \cdot \left( (H + h) \mb{u} \right) v_3 \, \dee \mb{x} &= 0.
% \end{align*}
% In this case the surface integrals drop out of the weak form as previously, but
% there are no boundary terms on the $h$-component weak form, so we just leave
% this as-is.

% Once again, the time-discretisation of this is:
% \begin{align*}
%   \int_\Omega
%   \left(\frac{ u_1^n - u_1^{n - 1}}{\Dt} \right) v_1
%   + \mb{u}^{n - 1} \cdot \nabla u_1^n \, v_1
%   + g h_x^n \, v_1
%   + C_D \frac{\lVert \mb{u}^{n - 1} \rVert_2}{H + h^n} u_1^n \, v_1
%   + \nu \nabla u_1^n \cdot \nabla v_1 \, \dee \mb{x} &= 0, \\
%   \int_\Omega \left( \frac{u_2^n - u_2^{n - 1}}{\Dt} \right) v_2
%   + \mb{u}^{n - 1} \cdot \nabla u_2^n \, v_2
%   + g h_y^n \, v_2
%   + C_D \frac{\lVert \mb{u}^{n - 1} \rVert_2}{H + h^n} u_2^n \, v_2
%   + \nu \nabla u_2^n \cdot \nabla v_2 \, \dee \mb{x} &= 0, \\
%   \int_\Omega \left( \frac{h^n - h^{n - 1}}{\Dt} \right) v_3
%   + \int_\Omega \nabla \cdot \left( (H + h^n) \mb{u}^n \right) v_3 \, \dee \mb{x} &= 0.
% \end{align*}


\subsection*{Specific weak form: Flow past a cylinder}

For the flow past a cylinder, we consider the rectangular domain
$\Omega = [0, d_x] \times [0, d_y]$, with a cylinder in the domain removed. This
gives us four boundaries for the rectangle: $(\Gamma_L, \Gamma_R, \Gamma_T,
\Gamma_B)$. There is one additional boundary for the surface, which we
call $\Gamma_{\mathrm{cyl}}$. The BC's are all Dirichlet BC's and are:
\[
  \mb{u} =
  \begin{cases}
    (u_i, 0) & \mb{x} \in \Gamma_L \cup \Gamma_R, \\
    (0, 0) & \mb{x} \in \Gamma_T \cup \Gamma_B \cup \Gamma_{\mathrm{cyl}}, \\
  \end{cases}
\]
So we have a specific inflow and outflow condition, and a no-slip condition on
the top and bottom walls, and on the cylinder. On the $h$ component, there are
no boundary conditions. For this simulation we integrate the divergence
term by parts. 
\begin{align*}
  \int_\Omega \partial_t \mb{u} \cdot \mb{v} \, \dee \mb{x}
  + \int_\Omega \left(\mb{u} \cdot \nabla \mb{u}\right) \cdot \mb{v} \, \dee \mb{x}
  + \int_\Omega g \nabla h \cdot \mb{v} \, \dee \mb{x}
  + \int_\Omega \nu \mb{T} : \nabla \mb{v} \, \dee \mb{x}
  + \int_\Omega C_D \frac{\lVert \mb{u} \rVert_2}{H + h} \mb{u} \cdot \mb{v} \, \dee \mb{x} = 0, \\
  \int_\Omega \partial_t h v
  - (H + h) \nabla v \cdot \mb{u} \dee \mb{x}
  + \int_{\Gamma_L} v (H + h) \mb{u} \cdot \mb{n} \, \dee s
  + \int_{\Gamma_R} v (H + h) \mb{u} \cdot \mb{n} \, \dee s = 0.
\end{align*}
As we prescribe all Dirichlet BC's for $\mb{u}$ vector fields, the boundary term
for the stress tensor drops out. Boundary terms from the continuity equation
which are not the inlet and outlet also drop out due to the no-slip conditions.

Which, upon completing a time discretisation, gives
\begin{align*}
  \int_\Omega \left(\frac{\mb{u}^n - \mb{u}^{n - 1}}{\Delta_t}\right) \cdot \mb{v} \, \dee \mb{x}
  + \int_\Omega \left(\mb{u}^n \cdot \nabla \mb{u}^n \right) \cdot \mb{v} \, \dee \mb{x} \\
  + \int_\Omega g \nabla h^n \cdot \mb{v} \, \dee \mb{x}
  + \int_\Omega \nu \mb{T}^n : \nabla \mb{v} \, \dee \mb{x}
  + \int_\Omega C_D \frac{\lVert \mb{u}^{n - 1} \rVert_2}{H + h^n} \mb{u}^n \cdot \mb{v} \, \dee \mb{x} &= 0, \\
  \int_\Omega \left(\frac{h^n - h^{n - 1}}{\Delta_t}\right) v - (H + h^n) \mb{u}^n \cdot \nabla v \dee \mb{x}
  + \int_{\Gamma_L} v (H + h^n) \mb{u}^n \cdot \mb{n} \, \dee s
  + \int_{\Gamma_R} v (H + h^n) \mb{u}^n \cdot \mb{n} \, \dee s &= 0.
\end{align*}

\subsubsection*{Large eddy simulation addition}

When solving the SWE in this setting we need to include an additional term on
the kinematic viscosity in order to model viscous effects which arise to eddies
in the solution domain, which are not resolved due to large grid sizes. In this
we use the Smagorinsky LES which adds in an additional viscosity coefficient,
$\nu_t$, to the viscosity coefficient $\nu$ in the equations above. This is
given by
\[
  \nu_t = (C_s \Delta_h)^2 \left| \mb{S} \right|,
\]
where $C_s$ is the Smagorinsky coefficient, $\Delta_h$ is the square root of the
area of each local element, and the modulus of the strain tensor is
\[
  \mb{S} = \frac{1}{2}\left( \nabla \mb{u} + \nabla \mb{u}^\top \right),
  \left| \mb{S} \right| = \sqrt{2 \sum_{i,j} \mb{S}_{ij}^2}.
\]
Note that this coefficient is proportional to the gradient of the velocity field
and the area of the elements (less refined meshes give higher values of $\nu_t$).





\bibliographystyle{dcu}
\bibliography{/Users/connor/Documents/LaTeX/master.bib}

\end{document}